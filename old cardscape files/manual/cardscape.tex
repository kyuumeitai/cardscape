\documentclass[a4paper, 11pt]{scrbook}
\usepackage[utf8]{inputenc}
\usepackage[colorlinks=true, linkcolor=blue]{hyperref}
\title{Cardscape - A web based development and publishing tool for collecting card games}
\author{The WTactics Devs} %May be changed later

\begin{document}
\maketitle
\tableofcontents

\section{License}
Cardscape is licensed under the Affero General Public License version 3%TODO Add link to license text

\chapter{Introduction}
Over the last two decades there have been several success stories on collaboration over the internet. Beginning with the development of the Linux kernel and not ending with collaborative efforts like Wikipedia, the internet has brought a new way of sharing and crafting to human kind.

The World Wide Web is the most used service on the internet today. It once started out as a service to distribute static HTML files and has been further developed to become the ``Web 2.0'' which uses dynamically generated content.

The fundamental part of the ``Web 2.0'' is collaboration. People from all over the world can create content and improve or extend existing content.

Cardscape is another one of these Web 2.0 tools. It allows the users to create concepts for their favorite collecting card games. These concepts may be further discussed by the game's community and finally be approved by the gamemakers.

\section{Installation}
Cardscape requires PHP 5 and MySQL 5. To install Cardscape edit the \texttt{config.php} file. All configuration directives will be explained in section \ref{configExplain}. CardScape comes with a \texttt{Card.php} file that is optimized for the WTactics card game. If you do not want to work with WTactics cards, you need to replace this file with something more appropriate.

Once you have made all the necessary adjustments, execute the \texttt{install.php} using your browser. This will set up the database and if everything works, the installation is complete.

\chapter{Using Cardscape}
Cardscape offers a variety of actions it can perform. These actions will be presented in the following sections:


\section{Cards and their properties}
Each card has several variables associated with it. The most apparrent properties are the card's name and its rule text. Depending on what kind of game Cardscape is used for, there may be further game-specific properties. However, there are also properties that are used internally by cardscape to keep track of cards and their relationships.

Cardscape distinguishes between official cards and cards that are in the so-called \emph{Card Development Area}. A card in the Card Development Area may later become an official card. So in fact, the official cards are a subset of all cards in the Card Development Area. But despite this redundancy, Cardscape associates different information with cards in both areas. The differences between these two areas will be discussed in the following sections.

\subsection{Card Development Area}\label{CardDevArea}
Most of the cards in the Card Development Area are not official cards. The Card Development Area is used to develop and discuss new cards. Another feature of the Card Development area is also the possibility to trace the ancestry of cards. You can start with a card and see which other cards are based on this card and you can also see if there is another card which is the predecessor of this card.

During the development process, a card may have one of the following statuses:

\begin{description}
 \item[Concept:] This is not a real card. It is used to propose a general concept that (other) users may take to create a card. If someone has a new idea for a new ability, he may create a concept and derive several cards as examples for this new ability from it.
 \item[New:] This card is new and there haven't been any comments on this card yet (comments from the card's creator do not count). Cards in this state may be changed by the author to any extend.
 \item[Discussed:] This card has at least one comment associated with it. Only minor edits will be possible (e.g. spelling errors) by the author. For greater changes a new draft must be made. %Use levenstein() to check impact of change?
 \item[Playtested:] The gamemaker crew has tested the card. It will sooner or later become official or rejected.
 \item[Official:] The card is now an official card in the game. A link to the official version of the card is provided.
 \item[Rejected:] The card has not been accepted as an official part of the game. The reasons for the rejection should be written down in the discussion thread for the card.
 \item[Superseded:] A newer card replaces this formerly official card.
\end{description}

\subsection{Official cards Area}
The official cards area contains only cards that are legal and well balanced. However, there is always a link to the same card in the Card Development Area if you are interested in the card's development history.

Each official card has a numerical ID. This ID is unique to this card as long as it's official. If another card supersedes this card, the new card will take this ID and the old card vanishes from the official cards area.

By looking at the highest ID you can also tell how many official cards a game does have.

In addition to the ID, an official card has a revision number counter telling how many times the ID of the card needed to be passed on to a newer (better) card. A high number signifies a rather problematic card that needed several adjustments to become balanced.

The revision number of a card is displayed directly after the card's ID, separated by a dot.

\section{Roles in Cardscape}
While the creation of new cards is a community driven process, there need to be some users with special privileges and competence to finally decide upon the acceptance of a card in the official set. Because of this, Cardscape uses the following roles to distinguish what a user can do:

\begin{description}
 \item[Guest:] Someone who in not logged in.
 \item[User:] Someone who has registered with the system.
 \item[Moderator:] Someone who has greater privileges to fight vandalism in the system.
 \item[Gamemaker:] Someone who is allowed to change a card's status manually.
 \item[Admin:] Keeps the site running and manages user roles.
\end{description}

\section{Searching for cards}
The user can search for cards using several criteria like the card's name but also numerical ranges like the cost of a card. These criteria may be combined with \texttt{AND} and \texttt{OR} clauses. Criteria may be inverted with the \texttt{NOT} clause.

Special searches like ``all cards by the following author'' are also possible.

Each search query can be saved as a bookmark in the browser since parameters are passed by HTTP GET.

There are different search options available for the Card Development Area and the official cards.

\section{Discussing Drafts}
There is a discussion thread for each card in the Card Development Area. There you can comment on new cards and tell others if you think this card is worthy adoption into the official set. If not, you can criticise and suggest further improvements to the card so that it becomes more balanced and/or interesting.

You can also use the discussions to post additional information on the card. For example, if you like to add epic background information to a card, the discussion thread is the right place for it.

The discussion thread serves as documentation for cards and their development process. Any player who wants to have more background information for a card can read the discussion thread of the card itself or it's predecessors.

Besides written comments there is also the possibility to rate a card with 1 to 5 stars.

\section{Commenting on official cards}
There is also the possibility to leave comments on cards in the official card area. But these comments should reflect the player's experiences with the card and should also be about combinations with other cards. It is also OK to leave comments on the imbalanced state of a card, but these comments should be rare.

Each official card may be rated for its popularity with 1 to 5 stars.
\section{Creating new card drafts}
There are two possibilities to create a new card draft: Use an existing card as inspiration/predecessor or start with a completely new card. If you choose the former, there will always be a link from your card to the inspiring/preceding card which can be interesting for other users.

The difference between a predecessor and an inspiration is the following: A predecessor will get replaced in the official set by its successor if the successor has the same name. An inspiration is just a variant of an existing card.

When creating a card, you cannot choose the name of an existing official card unless you explicitly use the card with the existing name as predecessor.

If you choose to use an existing card (no matter if official or not), you will see a form field where you can write your thoughts for the variant of the card down. These thoughts will be used as the first post in the discussion thread of the card.

Whenever a new card is created, its author and the date are recorded.

\section{Making cards official}
After a card has been playtested and discussed enough, any gamemaker may declare a card as official. As a matter of politeness, the gamemaker should not make any card official, if no other gamemakers have expressed their agreement for a card in the card's discussion thread.

Therefore the act of making a card official is by votes. But one doesn't need to wait for every gamemaker to approve. A few approval are sufficient.

\section{Using the messaging system}
There is the possibility to send messages between registered users of the system. This system should be used to recruit more moderators and gamemakers if necessary.


\chapter{Technical Details}
Technical stuff ahead!

\section{Third Party services}

Cardscape uses the following third party components to enhance user experience:
\subsection{Gravatar}
\href{http://www.gravatar.com/}{Gravatar} is used to display avatars in the discussion threads.

\subsection[reCaptcha]{reCaptcha}
\href{http://www.recaptcha.net}{reCaptcha} is used to prevent spam. The service is currently not available (HTTP 301) for unknown reasons.

\section{Coding Conventions}
Make beautiful code, not war. Code is documented with useful Doxygen-compatible comments.

\section{Card generation}
This is a very tricky part. There are individual modules for each card game because the layout between games may differ drastically. The code for rendering cards is located in the \texttt{Card.php} file. This file is optimized for the WTactics game. You need to replace this file if you want to use Cardscape with another game.

The approach for generating cards in WTactics is explained in the following subsection.

\subsection{WTactics}
Cards in the Card Development Area may have either a text as description of the image or an URL leading to an image that is already uploaded to the WTactics servers. In contrast all official cards must have an image associated with them.

Each image may overlap with the physical left and right borders of the card. There is also an extended border at the bottom that may be used for overlaps.

The card painting algorithm takes the associated image and aligns it with the right physical border and  the non-overlapping border at the bottom (The non-overlapping border is used to display the card's ID and revision number).

The image itself should only consist of any numbers of items/figures in the foreground and a transparent background. If the artist does not want his image to overlap with a specific border, he must leave a transparent border of equal size on the image.

The image painting algorithm will not resize cards. One reason for this is that if the artist leaves a transparent margin, the placement of the card will look bad in some cases.

If a card appears to have too much empty space on it, the user should increase the font size or write a longer flavour text.

Flavor text will be in italic.

When an image is chosen for a card, the user will write the rule- and flavortext directly onto the card using JavaScript. Whenever the text comes too close to the image, the user needs to insert a line break. If padding on the left side is required, the user may use space characters.

After the text is written on the card, the user will be able to preview the final card with the text drawn by PHP. Ideally JavaScript and PHP should behave identically but to make sure everything is laid out correctly, this preview is necessary.

\emph{We still need to discuss how unit/creature cards should be laid out.}

Card art should be referenceable from the image gallery (to be installed by snowdrop). That also means that the technical and legal difficulties with uploading images is not a concern of Cardscape. Reminder: The gallery should also allow uploads through external URLs.

But in any case there must be a link to the image gallery upload form from Cardscape.

\section{Login System}
Cardscape uses salted passwords and PHP Sessions. reCaptcha is used to prevent bots from registering.

\section{Config file}
Configuration will be done in a PHP file (\texttt{config.php}. It will include a big associative array.

\subsection{Card object}
Because each card game is different, each will need its own version of the \texttt{Card.php}. The \texttt{Card.php} should define two card classes: One for the cards in the Card Development Area and one for official cards.

With an class for cards, it is very convenient to cast an MySQL result to an object. Both classes should have a show() method to render the data to HTML with one line of code. There will be more methods to allow the creation of a completely new card and to edit an existing card.

Upon installation this file will be introspected to determine the required database layout.

\section{Ancestry tracking}
Each card in the drafts table has an id (which is also the primary key). Each card has also a foreign key which is identical to its ancestor's id. This way it is easy to find the ancestor and the successors of a card.

\section{Card translations}
Only official cards will support translations into other languages. It is too much work to translate draft cards. The translated versions of the cards will only differ from the English one by their rule- and flavortext and by the card's name. For right-to-left languages the card's art will be mirrored horizontally.

\appendix
\chapter{Reference}
\section{Configuration Directives in config.php}\label{configExplain}
Level values correspond to the role a user has. The following levels exist:
\begin{description}
 \item[0] Guest
 \item[1] Guest who has identified himself as a human by using reCaptcha
 \item[2] Registered user
 \item[3] Moderator
 \item[4] Gamemaker
 \item[5] Admin
\end{description}

\newcommand{\boolval}[0]{(\texttt{true}\textbar \texttt{false})}

\begin{description}
 \item[accept] \boolval - set to \texttt{true} if you accept Cardscape's license and have finished the configuration.
 \item[Database]
  \begin{description}
   \item[host] Name of the database host
   \item[user] Username for the database
   \item[pass] Password for the database
   \item[database] The database Cardscape should use
   \item[prefix] The prefix for tables. Default: wt\_
  \end{description}
 
 \item[auth]
  \begin{description}
   \item[salt] The salt used for the user's passwords in the database (md5 hash).
   \item[newdraft] Level - What kind of users should be allowed to create new drafts?
   \item[ratecard] Level - What kind of users should be allowed to rate a card?
   \item[commentcard] Level - What kind of users should be allowed to comment on a card?
   \item[registration] \boolval - enable registration of new users 
  \end{description} %TODO moar

\end{description}

\chapter{Symbols}
\section{WTactics}
The following symbols are used for WTactics.
\begin{itemize}
 \item Mark symbol
 \item faction logos
 \item gold symbol
 \item Threshold symbols for each fraction
\end{itemize}
These symbols should be referenced in the \texttt{Card.php} file.

\chapter{Version 1.1 Features}
\begin{itemize}
 \item Deck creation
 \item Statistics for decks
 \item Real player profiles
 \item Export of decks to other formats
 \item Admin frontend
\end{itemize}

\end{document}

